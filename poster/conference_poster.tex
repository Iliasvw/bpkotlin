%\title{LaTeX Portrait Poster Template}
%%%%%%%%%%%%%%%%%%%%%%%%%%%%%%%%%%%%%%%%%
% a0poster Portrait Poster
% LaTeX Template
% Version 1.0 (22/06/13)
%
% The a0poster class was created by:
% Gerlinde Kettl and Matthias Weiser (tex@kettl.de)
% 
% Adapter by Jens Buysse for Hogeschool Gent
% This template has been downloaded from:
% http://www.LaTeXTemplates.com
%
% License:
% CC BY-NC-SA 3.0 (http://creativecommons.org/licenses/by-nc-sa/3.0/)
%
%%%%%%%%%%%%%%%%%%%%%%%%%%%%%%%%%%%%%%%%%

%----------------------------------------------------------------------------------------
%	PACKAGES AND OTHER DOCUMENT CONFIGURATIONS
%----------------------------------------------------------------------------------------

\documentclass[a0,portrait]{a0poster}

\usepackage{multicol} % This is so we can have multiple columns of text side-by-side
\columnsep=100pt % This is the amount of white space between the columns in the poster
\columnseprule=3pt % This is the thickness of the black line between the columns in the poster

\usepackage[svgnames]{xcolor} % Specify colors by their 'svgnames', for a full list of all colors available see here: http://www.latextemplates.com/svgnames-colors

\usepackage{times} % Use the times font
%\usepackage{palatino} % Uncomment to use the Palatino font

\usepackage{graphicx} % Required for including images
\graphicspath{{figures/}} % Location of the graphics files
\usepackage{booktabs} % Top and bottom rules for table
\usepackage[font=small,labelfont=bf]{caption} % Required for specifying captions to tables and figures
\usepackage{amsfonts, amsmath, amsthm, amssymb} % For math fonts, symbols and environments
\usepackage{wrapfig} % Allows wrapping text around tables and figures
\usepackage[export]{adjustbox}
\usepackage{caption}
\usepackage{subcaption}


\begin{document}

%----------------------------------------------------------------------------------------
%	POSTER HEADER 
%----------------------------------------------------------------------------------------

% The header is divided into two boxes:
% The first is 75% wide and houses the title, subtitle, names, university/organization and contact information
% The second is 25% wide and houses a logo for your university/organization or a photo of you
% The widths of these boxes can be easily edited to accommodate your content as you see fit

\begin{minipage}[t]{0.75\linewidth}
\VeryHuge \color{HoGentAccent1} \textbf{Kotlin/Native, het nieuwe cross-platform framework voor de mobiele omgeving} \color{Black}\\ % Title
%\Huge\textit{Ondertitel (eventueel)}\\[2.4cm] % Subtitle
\huge \textbf{Van Wassenhove Ilias, Goossens Sander, Van Schoor Johan}\\[0.5cm] % Author(s)
\huge Hogeschool Gent, Valentin Vaerwyckweg 1, 9000 Gent\\[0.4cm] % University/organization
\Large \texttt{ilias.vanwassenhove.w9579@student.hogent.be} \\
\end{minipage}
%
\begin{minipage}[t]{0.25\linewidth}
\includegraphics[width=13cm,right]{figures/HG-woordmerk.jpg} 

\end{minipage}

\vspace{1cm} % A bit of extra whitespace between the header and poster content

%----------------------------------------------------------------------------------------

\begin{multicols}{2} % This is how many columns your poster will be broken into, a portrait poster is generally split into 2 columns

%----------------------------------------------------------------------------------------
%	ABSTRACT
%----------------------------------------------------------------------------------------

\color{HoGentAccent1} % Navy color for the abstract

\begin{abstract}
Nog niet zo lang geleden had men bij het bouwen van mobiele applicaties (Android, iOS en Windows) enkel en alleen de mogelijkheid om drie aparte applicaties te bouwen, native applicaties. Dit was voor vele bedrijven veel werk en  kostelijk. Als reactie hierop zijn de hybride frameworks uitgevonden, denk maar aan: React, Xamarin en Angular. Bij deze frameworks moest er maar éénmaal code geschreven worden en het framework zorgde ervoor dat de applicatie op elk besturingssysteem kon draaien. Er mag namelijk nog een framework toegevoegd worden aan het lijstje, namelijk Kotlin/Native.

\begin{center}\vspace{1cm}
	\includegraphics[width=1.0\linewidth]{figures/kotlin.png}
	\captionof{figure}{\color{HoGentAccent5}Officiele Kotlin logo}
\end{center}\vspace{1cm}

\end{abstract}
%----------------------------------------------------------------------------------------
%	INTRODUCTION
%----------------------------------------------------------------------------------------

\color{HoGentAccent1} 
\section*{Introductie}
\color{black}
\color{black}
Kotlin is een nieuwe programmeertaal die geintroduceerd werd in 2011 door JetBrains. Origineel is het een programmeertaal die draait op de Java Virtual Machine (JVM). Maar JetBrains is veel verder gegaan dan toestellen die een JVM kunnen draaien. Met Kotlin/Native hebben ze zich gericht tot alle platformen en besturingssystemen. In dit onderzoek zal onderzocht hoe JetBrains ervoor gezorgd heeft dat Kotlin code op ieder platform kan worden gebruikt (LLVM), hoe Kotlin/Native werkt en hoe het gebruikt kan worden voor cross-platform ontwikkeling.

\begin{center}\vspace{1cm}
	\includegraphics[width=1.0\linewidth]{figures/kn.png}
	\captionof{figure}{\color{HoGentAccent5}Kotlin/Native voor cross-platform applicatie ontwikkeling}
\end{center}\vspace{1cm}
%----------------------------------------------------------------------------------------
%	GEOLOGY
%----------------------------------------------------------------------------------------

\color{Black} % DarkSlateGray color for the rest of the content
\color{HoGentAccent1} 
\section*{Experimenten}
\color{black}
Voor dit onderzoek werd er praktisch aan de slag gegaan met Kotlin/Native. Er werd een kleine shopping applicatie gemaakt dat zowel Android als iOS ondersteunt. De gebruiker kan op deze applicatie een winkelmandje aanmaken, alle producten bekijken en eventueel de details van een product opvragen. Hij heeft ook de mogelijkheid om een product toe te voegen aan zijn winkelmandje en om zijn winkelmandje te bekijken. Door gebruik te maken van Kotlin/Native wordt alle domeinlogica van deze applicatie gedeeld over de verschillende ondersteunde platformen, dit zijnde iOS en Android.

\begin{center}\vspace{1cm}
	\includegraphics[width=0.8\linewidth]{figures/apple-android.jpeg}
	\captionof{figure}{\color{HoGentAccent5}Ondersteunde platformen van de proof-of-concept}
\end{center}\vspace{1cm}

\iffalse
\color{HoGentAccent1} 
\section*{Sectie met figuur}
\color{black}


\begin{center}\vspace{1cm}
\includegraphics[width=1.0\linewidth]{grail}
\captionof{figure}{\color{HoGentAccent5} He hasn't got shit all over him. The nose? Where'd you get the coconuts? What do you mean? We shall say 'Ni' again to you, if you do not appease us}
\end{center}\vspace{1cm}
\fi

%------------------------------------------------



\color{HoGentAccent1} 
\section*{Conclusies}
\color{black}
Kotlin/Native beschikt reeds over de capaciteiten om gebruikt te worden als een cross-platform framework. Echter zijn de mogelijkheden nog beperkt. Het framework is nog zeer jong en de huidige versie is slechts 0.6. Er is duidelijk eerst en vooral nood aan een stabiele versie waarmee ontwikkelaars aan de slag kunnen. Indien een applicatie over een grote en ingewikkelde domeinlogica beschikt, heeft dit framework zeker en vast een groot voordeel ten opzichte van native Android en iOS applicaties. Echter moeten de user interfaces per platform worden opgebouwd. Er is nog geen mogelijkheid om via Kotlin/Native een user interface op te bouwen. Dit kan zowel positief als negatief beschouwd worden. Enerzijds is er de mogelijkheid om verschillen aan te brengen in de user interface per platform, anderzijds moeten er dus verschillende interfaces gebouwd worden wat dubbel zoveel tijd en geld kost om te ontwikkelen.
%----------------------------------------------------------------------------------------
%	FORTHCOMING RESEARCH
%----------------------------------------------------------------------------------------
\color{HoGentAccent1} 
\section*{Toekomstig onderzoek}
\color{black}
Wat toekomstig onderzoek betreft in verband met Kotlin/Native is er nog veel mogelijk. Er kan onderzocht worden of er de mogelijkheid is om een plugin te ontwikkelen voor een IDE waardoor de opzet van een project mogelijk zou zijn via enkele klikken. Verder kan eventueel bekeken worden hoe men in de Kotlin code gebruik kan maken van enkele iOS protocollen en andere bibliotheken. Ten slotte is het aanspreken van hardware, zoals de camera, totaal nog niet onderzocht voor Kotlin/Native. Zoals te lezen is, is er nog veel ruimte tot onderzoek voor dit framework.

%----------------------------------------------------------------------------------------

\end{multicols}
\end{document}