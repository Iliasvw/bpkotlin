%%=============================================================================
%% Methodologie
%%=============================================================================

\chapter{Methodologie}
\label{ch:methodologie}

%% TODO: Hoe ben je te werk gegaan? Verdeel je onderzoek in grote fasen, en
%% licht in elke fase toe welke stappen je gevolgd hebt. Verantwoord waarom je
%% op deze manier te werk gegaan bent. Je moet kunnen aantonen dat je de best
%% mogelijke manier toegepast hebt om een antwoord te vinden op de
%% onderzoeksvraag.

\section{Opzoeken informatie over Kotlin}
Nadat het bachelorproefvoorstel werd goedgekeurd werd er onmiddellijk aan de slag gegaan. De eerste stap voor deze bachelorproef was het opzoeken van informatie rond Kotlin. Het was belangrijk om alle mogelijkheden van Kotlin goed in kaart te brengen. Tijdens deze studie werd ontdekt dat Kotlin reeds veel mogelijkheden heeft. Het kan momenteel gebruikt worden voor server-side, JavaScript en Android applicatieontwikkeling. Ondertussen werd er geëxperimenteerd met enkele Kotlin projecten om gewoon te worden aan de Kotlin syntax.

Het doel van deze bachelorproef was het onderzoeken of Kotlin reeds enkele mogelijkheden had om gebruikt te worden voor cross-platform applicatieontwikkeling. Hierbij werd duidelijk dat JetBrains bezig was met het ontwikkelen van een framework dat gebruikt kon werden voor cross-platform applicaties te ontwikkelen. Echter was er op de website van JetBrains weinig informatie te vinden over dit framework. Meer onderzoek was dus nodig.

\section{Opzoeken informatie over Kotlin/Native}
\label{sec:infokn}
Nadat er kennis werd gemaakt met Kotlin/Native, was het duidelijk dat er meer informatie nodig was om zomaar met dit framework aan de slag te gaan. Eerst en vooral werd de beknopte informatie over Kotlin/Native op de website van Kotlin doorgenomen. Op de website werd duidelijk dat Kotlin/Native gebruik maakte van een speciale compiler, namelijk LLVM. Hierover moest informatie opgezocht worden. Verder was er op het internet, behalve de algemene uitleg over Kotlin/Native, weinig nieuwe informatie te vinden en werd het duidelijk dat de echte werking aan de hand van enkele voorbeeldprojecten van JetBrains duidelijk moest worden.

\section{Bekijken voorbeeldprojecten JetBrains}
De volgende stap was het bekijken, onderzoeken en analyseren van de voorbeeldprojecten van JetBrains. Er waren verschillende voorbeelden aanwezig op de GitHub respository van JetBrains. Een voorbeeld hiervan was de Calculator applicatie. Een cross-platform applicatie waarbij de domeinlogica een rekenmachine was. Deze werd gedeeld tussen de verschillende platformen, zijde Android en iOS. De opzet van een Kotlin/Native project was helemaal niet duidelijk. De enige mogelijkheid om de werking van een project duidelijk te begrijpen was het kopiëren van een project en hierin aanpassingen te maken. Zo werd duidelijk wat de effecten waren van deze aanpassingen en kon de werking van bepaalde elementen achterhaald worden. Naast de voorbeeldprojecten van JetBrains werd er een repository gevonden op GitHub, van een developer genaamd \textcite{AlbertGao}. Deze persoon heeft zelf ook een Kotlin/Native project opgezet, gericht op Android en iOS.

\section{Praktische uitwerking Kotlin/Native}
Daarna was het tijd om de grote stap te zetten en volledig zelfstandig een Kotlin/Native project op te zetten. De volledige mappenstructuur van een voorbeeldproject werd overgenomen van de repository van \textcite{AlbertGao}. Na genoeg projecten onderzocht te hebben was het al iets gemakkelijker om een project vanaf nul op te zetten. Bij het opzetten werd duidelijk dat Gradle voortdurend gebruikt zou worden aangezien de volledige opzet van een project en het gebruik van de Kotlin/Native plug-in momenteel moet gebeuren via Gradle. Naarmate het project vorderde werd de werking van Gradle duidelijker en was het makkelijker om dit build systeem te gebruiken. Door het zelf schrijven van Gradle scripts werd de werking van Kotlin/Native ook duidelijker. Indien er iets ontbrak in de scripts, werden er nuttige foutboodschappen getoond in de console door Kotlin/Native zelf.

\section{Literatuurstudie LLVM-compiler}
Parallel met het onderzoeken van Kotlin/Native werd de compiler van dit framework bestudeerd. In sectie \ref{sec:infokn} werd geconstateerd dat Kotlin/Native de LLVM-compiler gebruikt. Na opzoekwerk op het internet bleek dat er niet bijzonder veel specifieke informatie te vinden was over deze compiler. Echter bleek dat er op het internet een aosabook \autocite{aosa}, 'The Architecture of Open Source Applications', bestond. Hierin werd de volledige werking van een eenvoudige compiler (voor één programmeertaal), een meer geavanceerde compiler (voor meerdere programmeertalen) en de LLVM-compiler volledig uitgelegd en het verschil tussen LLVM en een standaard compiler werd aangetoond. Voor dit onderzoek was deze documentatie een zeer grote hulp.

%Vooraleer er aan de slag kon worden gegaan met het uittesten van Kotlin/Native, of het analyseren van voorbeeldprojecten van JetBrains, was het belangrijk om te onderzoeken hoe het nu komt dat Kotlin/Native op verschillende platformen kan draaien. Kotlin is oorspronkelijk, net zoals Java, een programmeertaal die gebruik maakt van de Java Virtual Machine. Het is daarom nuttig te onderzoeken hoe het mogelijk is gemaakt om Kotlin cross-platform te gebruiken. Dit is gebeurd aan de hand van een literatuurstudie. Het was niet nodig om dit praktisch te doen aangezien dit geen meerwaarde zou bieden aan deze bachelorproef. Kotlin/Native maakt gebruik van de LLVM compiler. Deze literatuurstudie is uitgevoerd aan de hand van de aosabook van LLVM, \textcite{aosa}, een zeer uitgebreide documentatie. 

%Daarna werden de voorbeeldprojecten van JetBrains, die gebruik maken van het Kotlin/Native framework, geanalyseerd om de werking van dit framework te achterhalen. Hierbij was het de bedoeling om te onderzoeken hoe het mogelijk was, om in een zeer vroeg stadium van Kotlin/Native, reeds aan de slag te gaan met dit framework. Dit is begonnen met het downloaden van de code van de repositories van Kotlin/Native en de Kotlin multiplatform projects. Dit was niet gemakkelijk. Vaak waren er problemen met verkeerde versies of programma's die ontbraken op de computer. Na enige tijd kon er aan de slag worden gegaan met de projecten. In eerste instantie werden er dingen gewijzigd in het originele project van Kotlin/Native zelf om te kijken hoe alle modules, klassen, gradle bestanden werkten en gelinkt waren.

%Eenmaal de werking van Kotlin Native duidelijk was, was het de bedoeling om zelf aan de slag te gaan met dit framework en een kleine cross-platform applicatie te schrijven. Hierbij werd er eerst een Kotlin/Native project opgezet, wat toch enige tijd in beslag heeft genomen. Eens het project was opgezet kon de applicatie gebouwd worden. Het prototype dat is gemaakt is een kleine shopping cart applicatie. Aan de hand van de ervaringen die zijn opgedaan tijdens het opzetten en ontwikkelen van het project kon ik de huidige mogelijkheden, beperkingen en conclusie formuleren.

