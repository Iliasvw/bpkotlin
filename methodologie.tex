%%=============================================================================
%% Methodologie
%%=============================================================================

\chapter{Methodologie}
\label{ch:methodologie}

%% TODO: Hoe ben je te werk gegaan? Verdeel je onderzoek in grote fasen, en
%% licht in elke fase toe welke stappen je gevolgd hebt. Verantwoord waarom je
%% op deze manier te werk gegaan bent. Je moet kunnen aantonen dat je de best
%% mogelijke manier toegepast hebt om een antwoord te vinden op de
%% onderzoeksvraag.

Vooraleer er aan de slag kon worden gegaan met het uittesten van Kotlin Native, of het analyseren van voorbeeldprojecten van JetBrains, was het belangrijk om te onderzoeken hoe het nu komt dat Kotlin Native op verschillende platformen kan draaien. Kotlin is oorspronkelijk, net zoals Java, een programmeertaal die gebruik maakt van de Java Virtual Machine. Het is daarom nuttig te onderzoeken hoe het mogelijk is gemaakt om Kotlin cross-platform te gebruiken. Dit is gebeurd aan de hand van een literatuurstudie. Het was niet nodig om dit praktisch te doen aangezien dit geen meerwaarde zou bieden aan deze bachelorproef. Kotlin/Native maakt gebruik van de LLVM compiler. Deze literatuurstudie is uitgevoerd aan de hand van de aosabook van LLVM, \textcite{aosa}, een zeer uitgebreide documentatie. 

Daarna werden de voorbeeldprojecten van JetBrains, geschreven in Kotlin/Native, geanalyseerd om de werking van dit framework te achterhalen. Hierbij was het de bedoeling om te onderzoeken hoe het mogelijk was, om in een zeer vroeg stadium van Kotlin/Native, reeds aan de slag te gaan met dit framework. Dit is begonnen met het downloaden van de code van de repositories van Kotlin/Native en de Kotlin multiplatform projects. Dit was niet gemakkelijk. Vaak waren er problemen met verkeerde versies of programma's die ontbraken op de computer. Na enige tijd kon er aan de slag worden gegaan met de projecten. In eerste instantie werden er dingen gewijzigd in het originele project van Kotlin/Native zelf om te kijken hoe alle modules, klassen, gradle bestanden werkten en gelinkt waren.

Eenmaal de werking van Kotlin Native duidelijk was, was het de bedoeling om zelf aan de slag te gaan met dit framework en een kleine cross-platform applicatie te schrijven. Hierbij werd er eerst een Kotlin/Native project opgezet, wat toch enige tijd in beslag heeft genomen. Eens het project was opgezet kon de applicatie gebouwd worden. Het prototype dat is gemaakt is een kleine shopping cart applicatie.

