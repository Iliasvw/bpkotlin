%%=============================================================================
%% Inleiding
%%=============================================================================

\chapter*{Inleiding}
\label{ch:inleiding}

%De inleiding moet de lezer net genoeg informatie verschaffen om het onderwerp te begrijpen en in te zien waarom de onderzoeksvraag de moeite waard is om te onderzoeken. In de inleiding ga je literatuurverwijzingen beperken, zodat de tekst vlot leesbaar blijft. Je kan de inleiding verder onderverdelen in secties als dit de tekst verduidelijkt. Zaken die aan bod kunnen komen in de inleiding~\autocite{Pollefliet2011}:


%\begin{itemize}
 % \item context, achtergrond
  %\item afbakenen van het onderwerp
  %\item verantwoording van het onderwerp, methodologie
  %\item probleemstelling
  %\item onderzoeksdoelstelling
  %\item onderzoeksvraag
  %\item \ldots
%\end{itemize}

 De ontwikkeling van mobiele applicaties is een zeer belangrijke niche in de IT. Zo kunnen we drie verschillende besturingssystemen onderscheiden bij mobiele applicaties: Android, iOS en Windows. Die laatste wordt steeds minder en minder gebruikt. Volgens een artikel van \textcite{Statista2018}, een portaal voor statistieken en onderzoek, heeft Android een marktaandeel van 85\%. Dit wil dus zeggen dat 85\% van de mensen die een smartphone heeft, een toestel is met Android. 
 \newline
 \newline
 De mobiele applicaties die draaien op Android werden allemaal geschreven in Java. Maar sinds kort is er een nieuwe mogelijkheid, Kotlin. Kotlin is een programmeertaal die ontworpen is door JetBrains in 2011. Sinds 2017 biedt Google volledige ondersteuning om Android applicaties te programmeren in Kotlin. 

\section{Probleemstelling}
\label{sec:probleemstelling}
Vanuit dit onderzoek zal blijken hoe goed Kotlin is als cross-platform ontwikkelingstaal en wat de mogelijkheden zijn van Kotlin om te programmeren voor verschillende platformen. Deze vraag kwam vanuit mijn stagebedrijf (Endare BVBA). Een developer bij Endare is reeds bezig met het aanleren van Kotlin en heeft reeds enkele Android projecten gedaan met Kotlin. Bij Endare wordt er zeer veel aan cross-platform ontwikkeling gedaan. Zo heeft een mede-stagiair tijdens zijn stage bij Endare gewerkt met React Native. Een framework waarbij je via één codebase native applicaties kan schrijven voor iOS en Android. Tegenwoordig worden worden er nog maar weinig native applicaties gebouwd. Waarom? Voor veel bedrijven zijn native applicaties een zéér hoge kost. Je hebt enerzijds twee keer een kost om de applicaties te bouwen, anderszijds heb je twee keer een kost om de applicaties te onderhouden of eventueel uit te breiden. 
\newline
\newline
Tegenwoordig heb je meer dan genoeg keuze om hybride applicaties te maken. Zo is er:
\begin{itemize}
	\item Ionic
	\item Cordova
	\item React Native
	\item Xamarin
\end{itemize}
En ga zo maar door... Met Kotlin Native komt er een nieuw framework zich aanbieden. Dit framework is nog zeer jong en er bestaat bijna geen documentatie. Ontwikkelaars die dit framework willen gebruiken moeten momenteel zelf de werking van Kotlin Native proberen achterhalen. Daarom is het belangrijk om te bestuderen of Kotlin een mogelijkheid biedt tot een nieuwe cross-platform programmeertaal, hoe de compiler van Kotlin ervoor kan zorgen dat het op meerdere platformen kan draaien en hoe men nu juist gebruik kan maken van Kotlin Native. 
\newline
\newline
Voor het stagebedrijf, Endare, heeft deze bachelorproef een grote meerwaarde aangezien zij iedere dag met cross-platform frameworks aan de slag moeten.

%Uit je probleemstelling moet duidelijk zijn dat je onderzoek een meerwaarde heeft voor een concrete doelgroep. De doelgroep moet goed gedefinieerd en afgelijnd zijn. Doelgroepen als ``bedrijven,'' ``KMO's,'' systeembeheerders, enz.~zijn nog te vaag. Als je een lijstje kan maken van de personen/organisaties die een meerwaarde zullen vinden in deze bachelorproef (dit is eigenlijk je steekproefkader), dan is dat een indicatie dat de doelgroep goed gedefinieerd is. Dit kan een enkel bedrijf zijn of zelfs één persoon (je co-promotor/opdrachtgever).

\section{Onderzoeksvraag}
\label{sec:onderzoeksvraag}
De onderzoeksvragen voor deze bachelorproef zijn: 
\begin{itemize}
	\item Hoe zorgt de Kotlin compiler ervoor dat Kotlin op verschillende platformen kan gebruikt worden?
	\item Wat is de werking van Kotlin Native?
	\item In hoeverre kan Kotlin gebruikt worden voor cross-platform applicatieontwikkeling?
\end{itemize}
%Wees zo concreet mogelijk bij het formuleren van je onderzoeksvraag. Een onderzoeksvraag is trouwens iets waar nog niemand op dit moment een antwoord heeft (voor zover je kan nagaan). Het opzoeken van bestaande informatie (bv. ``welke tools bestaan er voor deze toepassing?'') is dus geen onderzoeksvraag. Je kan de onderzoeksvraag verder specifiëren in deelvragen. Bv.~als je onderzoek gaat over performantiemetingen, dan 

\section{Onderzoeksdoelstelling}
\label{sec:onderzoeksdoelstelling}
Voor deze bachelorproef zijn er verschillende criteria van succes:
\begin{itemize}
	\item De werking van de Kotlin compiler documenteren
	\item Goede documentatie opstellen over de werking van Kotlin Native
	\item Bepalen in hoeverre Kotlin gebruikt kan worden als cross-platform programmeertaal
\end{itemize}
%Wat is het beoogde resultaat van je bachelorproef? Wat zijn de criteria voor succes? Beschrijf die zo concreet mogelijk.

\section{Opzet van deze bachelorproef}
\label{sec:opzet-bachelorproef}
% Het is gebruikelijk aan het einde van de inleiding een overzicht te
% geven van de opbouw van de rest van de tekst. Deze sectie bevat al een aanzet
% die je kan aanvullen/aanpassen in functie van je eigen tekst.

De rest van deze bachelorproef is als volgt opgebouwd:

In Hoofdstuk~\ref{ch:stand-van-zaken} wordt een overzicht gegeven van de stand van zaken binnen het onderzoeksdomein, op basis van een literatuurstudie.

In Hoofdstuk~\ref{ch:methodologie} wordt de methodologie toegelicht en worden de gebruikte onderzoekstechnieken besproken om een antwoord te kunnen formuleren op de onderzoeksvragen.

In Hoofdstuk~\ref{ch:compiler} zal onderzocht worden, via een literatuurstudie, wat de werking is van de Kotlin compiler. Hoe ervoor kan gezorgd worden dat Kotlin ook op apparaten zonder een JVM kan draaien.

In Hoofdstuk~\ref{ch:kotlinnative} zal Kotlin Native bestudeerd worden. Hierin zal de werking van Kotlin Native gedocumenteerd worden.

In Hoofdstuk~\ref{ch:conclusie}, tenslotte, wordt de conclusie gegeven en een antwoord geformuleerd op de onderzoeksvragen. Daarbij wordt ook een aanzet gegeven voor toekomstig onderzoek binnen dit domein.

