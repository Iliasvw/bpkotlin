%%=============================================================================
%% Voorwoord
%%=============================================================================

\chapter*{Woord vooraf}
\label{ch:voorwoord}

Graag neem ik even de tijd om mijn onderwerpkeuze te motiveren.

Ik heb dit onderwerp gekozen omdat ik een grote passie heb voor programmeren. Tijdens het eerste semester van het derde academiejaar toegepaste informatica heb ik kennis mogen maken met Android en iOS, het ontwikkelen van mobiele (native) applicaties. Hierbij heb ik ontdekt dat dit de richting is waarin ik mij wens te specialiseren, namelijk mobiele applicaties. Tijdens de lessen Android werd er af en toe eens gediscussieerd waarom we nog steeds Java gebruikten in plaats van Kotlin. Ik had persoonlijk nog niet veel van Kotlin gehoord maar toen is mijn interesse ontstaan. Ik ben dan onmiddellijk begonnen met het opzoeken van informatie over deze programmeertaal. Hierbij ontdekte ik dat reeds heel wat bekende applicaties Kotlin gebruikten, voor de ontwikkeling van hun Android applicatie. Een aantal voorbeelden hiervan zijn Trello, Pinterest en Evernote. Hieruit kon ik constateren dat Kotlin reeds sterk gebruikt werd. De verschillende onderzoeken over Kotlin op het internet, waarin Kotlin vaak als een innovatieve taal werd voorgesteld, hebben er alleen voor gezorgd dat mijn interesse in deze programmeertaal groeide.

Toen het tijd was om een bachelorproefvoorstel in te sturen wou ik zeker en vast iets onderzoeken wat te maken heeft met mobiele applicaties. Niet veel later stuurde mijn stagebedrijf een bachelorproefonderwerp door over Kotlin en cross-platform applicaties. De keuze was snel gemaakt aangezien ik reeds een grote interesse in Kotlin had. 

Ik heb door deze bachelorproef kennis gemaakt met een nieuw cross-platform framework en ik heb Kotlin leren gebruiken voor Android development. Persoonlijk was dit een zeer interessant onderwerp om te onderzoeken en heb ik met veel plezier aan dit onderzoek gewerkt.

Tenslotte wil ik nog even de tijd nemen om enkele personen te bedanken.

Ten eerste wil ik mijn promotor, Johan Van Schoor, bedanken voor de hulp, raad en opbouwende commentaar die ik gekregen heb bij het opstellen van deze bachelorproef. Mede dankzij zijn begeleiding ben ik tot een mooi resultaat gekomen.

Ten tweede zou ik graag mijn co-promotor, Sander Goossens, willen bedanken, wie ook mijn stagementor was. Hij stond altijd paraat om mijn vragen te beantwoorden en indien ik hulp nodig had was het nooit een probleem om even uitleg te geven.

Tenslotte zou ik iedereen waarmee ik heb samengewerkt willen bedanken voor de vlotte samenwerking. Er werd steeds samengewerkt op een zeer toffe en ontspannen manier waarbij iedereen gerespecteerd werd.

Hierdoor kan ik met een voldaan gevoel deze bachelorproef afgeven. Ik wens u veel plezier tijdens het lezen van mijn werkstuk.

Ilias Van Wassenhove

%% TODO:
%% Het voorwoord is het enige deel van de bachelorproef waar je vanuit je
%% eigen standpunt (``ik-vorm'') mag schrijven. Je kan hier bv. motiveren
%% waarom jij het onderwerp wil bespreken.
%% Vergeet ook niet te bedanken wie je geholpen/gesteund/... heeft

