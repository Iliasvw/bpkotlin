%%=============================================================================
%% Samenvatting
%%=============================================================================
% thesis) synthese van het document.
%
% Deze aspecten moeten zeker aan bod komen:
% - Context: waarom is dit werk belangrijk?
% - Nood: waarom moest dit onderzocht worden?
% - Taak: wat heb je precies gedaan?
% - Object: wat staat in dit document geschreven?
% - Resultaat: wat was het resultaat?
% - Conclusie: wat is/zijn de belangrijkste conclusie(s)?
% - Perspectief: blijven er nog vragen open die in de toekomst nog kunnen
%    onderzocht worden? Wat is een mogelijk vervolg voor jouw onderzoek?
%
% LET OP! Een samenvatting is GEEN voorwoord!

%%---------- Nederlandse samenvatting -----------------------------------------
%
% TODO: Als je je bachelorproef in het Engels schrijft, moet je eerst een
% Nederlandse samenvatting invoegen. Haal daarvoor onderstaande code uit
% commentaar.
% Wie zijn bachelorproef in het Nederlands schrijft, kan dit negeren, de inhoud
% wordt niet in het document ingevoegd.

\IfLanguageName{english}{%
\selectlanguage{dutch}
\chapter*{Samenvatting}
\lipsum[1-4]
\selectlanguage{english}
}{}

%%---------- Samenvatting -----------------------------------------------------
% De samenvatting in de hoofdtaal van het document

\chapter*{\IfLanguageName{dutch}{Samenvatting}{Abstract}}

Nog niet zo lang geleden had men bij het bouwen van mobiele applicaties (Android, iOS en Windows) enkel en alleen de mogelijkheid om drie aparte applicaties te bouwen, native applicaties. Dit vergde zeer veel werk en was heel kostelijk voor veel bedrijven. Als reactie hierop zijn de hybride frameworks uitgevonden, denk maar aan: React, Xamarin en Ionic met Angular. Bij deze frameworks moest er maar eenmalig code geschreven worden, de user interface was hetzelfde voor alle platformen en het framework zorgde er voor dat je applicatie op elk besturingssysteem kon draaien. We mogen misschien nog een framework toevoegen aan het lijstje van hybride frameworks, namelijk Kotlin/Native.

Kotlin is een nieuwe programmeertaal die geïntroduceerd werd in 2011 door JetBrains. Origineel is het een programmeertaal die draait op de JVM\footnote{Java Virtual Machine}, net zoals Java. Maar JetBrains is veel verder gegaan dan toestellen die een JVM kunnen draaien. Met Kotlin/Native hebben ze zich gericht tot alle platformen en besturingssystemen. 

%In dit onderzoek zal onderzocht worden hoe het komt dat Kotlin/Native op alle platformen en besturingssystemen kan draaien en hoe Kotlin/Native nu precies werkt. 

Maar waarom is het nu belangrijk om Kotlin/Native gedetailleerd te gaan bestuderen? Kotlin/Native is nog heel jong en er zijn al heel wat ontwikkelaars die reeds Kotlin gebruiken voor verschillende doeleinden. Zij staan te springen om aan de slag te gaan met Kotlin/Native. De enige beperking die zij momenteel ervaren is documentatie. Van dit onderzoek wordt verwacht dat door het gedetailleerd bestuderen van Kotlin/Native een mooie documentatie kan opgeleverd worden over de werking van dit framework waar iedere ontwikkelaar met aan de slag kan gaan.

In dit onderzoek werd de werking van de LLVM compiler en Kotlin/Native onderzocht. Daarna werd er aan de hand van het onderzoek rond Kotlin/Native een kleine proof-of-concept opgesteld waarbij duidelijk moest worden wat de huidige mogelijkheden zijn van dit framework.

Het resultaat van dit onderzoek is enerzijds een documentatie over de werking van de LLVM compiler en Kotlin/Native, anderzijds een uitgewerkt proof-of-concept aan de hand van Kotlin/Native.

De conclusie die uit dit onderzoek kan worden getrokken is dat Kotlin/Native nog heel jong is en de mogelijkheden nog zeer beperkt zijn. Momenteel is JetBrains bezig met het verder uitwerken van versie 0.6 en er is eigenlijk nog geen release versie beschikbaar. Het is wel reeds mogelijk om aan de slag te gaan met dit framework maar het opzetten van een project is niet gemakkelijk. Er is geen officiële documentatie beschikbaar over de opzet van een Kotlin/Native project. Maar aan de hand van de proof-of-concept kan besloten worden dat Kotlin/Native momenteel wel reeds gebruikt kan worden om business logica te delen over verschillende platformen. Het is reeds mogelijk om Kotlin code te gebruiken voor iOS.

Dit framework heeft zeker en vast zijn troeven. Applicaties die een zeer uitgebreide domeinlogica hebben zoals bijvoorbeeld een webshop, hebben zeker voordeel indien deze applicaties Kotlin/Native gebruiken. De domeinlogica kan gedeeld worden over de verschillende platformen, waardoor deze maar één maal moet worden geschreven, en de user interface kan per platform opgebouwd worden. 

Er wordt sterk gewerkt aan Kotlin/Native en dat zal in de toekomst niet veranderen. Er is nog veel werk aan de winkel, de opzet van een project is momenteel heel omslachtig en hiervoor zal JetBrains toch een oplossing moeten vinden. De functionaliteiten zullen ook verder uitgebreid moeten worden. Zal er bijvoorbeeld in de toekomst de mogelijkheid zijn om via Kotlin de camera te openen, waardoor men specifiek per platform moet programmeren?