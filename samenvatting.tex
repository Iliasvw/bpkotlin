%%=============================================================================
%% Samenvatting
%%=============================================================================
% thesis) synthese van het document.
%
% Deze aspecten moeten zeker aan bod komen:
% - Context: waarom is dit werk belangrijk?
% - Nood: waarom moest dit onderzocht worden?
% - Taak: wat heb je precies gedaan?
% - Object: wat staat in dit document geschreven?
% - Resultaat: wat was het resultaat?
% - Conclusie: wat is/zijn de belangrijkste conclusie(s)?
% - Perspectief: blijven er nog vragen open die in de toekomst nog kunnen
%    onderzocht worden? Wat is een mogelijk vervolg voor jouw onderzoek?
%
% LET OP! Een samenvatting is GEEN voorwoord!

%%---------- Nederlandse samenvatting -----------------------------------------
%
% TODO: Als je je bachelorproef in het Engels schrijft, moet je eerst een
% Nederlandse samenvatting invoegen. Haal daarvoor onderstaande code uit
% commentaar.
% Wie zijn bachelorproef in het Nederlands schrijft, kan dit negeren, de inhoud
% wordt niet in het document ingevoegd.

\IfLanguageName{english}{%
\selectlanguage{dutch}
\chapter*{Samenvatting}
\lipsum[1-4]
\selectlanguage{english}
}{}

%%---------- Samenvatting -----------------------------------------------------
% De samenvatting in de hoofdtaal van het document

\chapter*{\IfLanguageName{dutch}{Samenvatting}{Abstract}}

Nog niet zo lang geleden had men bij het bouwen van mobiele applicaties (Android, iOS en Windows) enkel en alleen de mogelijkheid om drie aparte applicaties te bouwen, namelijk native applicaties. Dit vergde veel werk en was heel kostelijk voor veel bedrijven. Enerzijds is er per platform de tijd en kost om de applicatie te ontwikkelen, anderzijds is er de kost om deze applicaties uit te breiden of te onderhouden. Als reactie hierop zijn de cross-platform frameworks uitgevonden, denk maar aan: React, Xamarin en Ionic met Angular. Bij deze frameworks moet er maar éénmalig code geschreven worden, de user interface is hetzelfde voor alle platformen en het framework zorgt ervoor dat de applicatie op elk besturingssysteem kan draaien. Een kleine nuance, origineel had Xamarin niet de mogelijkheid om één user interface te ontwikkelen die hetzelfde was voor alle platformen. Er mag misschien nog een framework toegevoegd worden aan het lijstje van frameworks om cross-platform applicaties te ontwikkelen, namelijk Kotlin/Native.

Kotlin is een nieuwe programmeertaal die geïntroduceerd werd in 2011 door JetBrains. Origineel is het een programmeertaal die draait op de JVM\footnote{Java Virtual Machine}, net zoals Java. Maar JetBrains is veel verder gegaan dan toestellen die een JVM kunnen draaien. Met Kotlin/Native hebben ze zich gericht tot alle platformen en besturingssystemen. 

%In dit onderzoek zal onderzocht worden hoe het komt dat Kotlin/Native op alle platformen en besturingssystemen kan draaien en hoe Kotlin/Native nu precies werkt. 

Maar waarom is het nu belangrijk om Kotlin/Native gedetailleerd te gaan bestuderen? Kotlin/Native is nog heel jong en er zijn al heel wat ontwikkelaars die reeds Kotlin gebruiken voor verschillende doeleinden. Deze ontwikkelaars staan te springen om aan de slag te gaan met Kotlin/Native. Met Kotlin spreken we over de programmeertaal, terwijl met Kotlin/Native het framework voor cross-platform applicatieontwikkeling bedoeld wordt dat gebruik maakt van de programmeertaal Kotlin, zie sectie \ref{sec:differenceKotlinAndNative} voor verdere uitleg. De enige beperking die zij momenteel ervaren is documentatie. Van dit onderzoek wordt verwacht dat door het gedetailleerd bestuderen van Kotlin/Native een mooie documentatie kan opgeleverd worden over de werking van dit framework waar iedere ontwikkelaar mee aan de slag kan gaan.

In dit onderzoek werd de werking van de LLVM-compiler en Kotlin/Native onderzocht. Voor dit onderzoek is het belangrijk om de LLVM te onderzoeken aangezien deze ervoor zorgt dat Kotlin omgezet wordt naar platformonafhankelijke uitvoerbare bestanden waardoor Kotlin op alle platformen kan gebruikt worden. Daarna werd er aan de hand van het onderzoek rond Kotlin/Native een kleine proof-of-concept opgesteld waarbij duidelijk moest worden wat de huidige mogelijkheden en beperkingen zijn van dit framework.

Het resultaat van dit onderzoek is enerzijds een documentatie over de werking van de LLVM-compiler en Kotlin/Native, anderzijds een uitgewerkt proof-of-concept aan de hand van dit framework.

De conclusie die uit dit onderzoek kan worden getrokken is dat Kotlin/Native nog heel jong is en de mogelijkheden nog zeer beperkt zijn. Momenteel is JetBrains bezig met het verder uitwerken van versie 0.6 en er is eigenlijk nog geen release versie beschikbaar. Het is wel reeds mogelijk om aan de slag te gaan met dit framework maar het opzetten van een project is niet gemakkelijk. Dit aangezien er momenteel geen plug-in bestaat die geïnstalleerd kan worden in een IDE zoals IntelliJ en er is geen officiële documentatie beschikbaar over de opzet van een Kotlin/Native project. Maar aan de hand van de proof-of-concept kan besloten worden dat Kotlin/Native momenteel wel reeds gebruikt kan worden om domeinlogica te delen tussen verschillende platformen. Het is dus mogelijk om domeinlogica, die geschreven is in Kotlin, te gebruiken voor iOS.

Dit framework heeft zeker en vast zijn troeven. Applicaties die een zeer uitgebreide domeinlogica hebben zoals bijvoorbeeld een webshop, hebben zeker voordeel indien deze applicaties Kotlin/Native gebruiken. De domeinlogica kan gedeeld worden over de verschillende platformen, waardoor deze maar éénmaal moet worden geschreven. De user interface kan per platform opgebouwd worden. 

Er wordt sterk gewerkt aan Kotlin/Native en dat zal in de toekomst niet veranderen. Er is nog veel werk aan de winkel, de opzet van een project is momenteel heel omslachtig en hiervoor zal JetBrains toch een oplossing moeten vinden. De functionaliteiten zullen ook verder uitgebreid moeten worden. Zal er bijvoorbeeld in de toekomst de mogelijkheid zijn om via Kotlin de camera te openen, waardoor dit niet specifiek per platform moet worden geprogrammeerd?