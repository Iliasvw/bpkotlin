%%=============================================================================
%% Conclusie
%%=============================================================================

\chapter{Conclusie}
\label{ch:conclusie}

%% TODO: Trek een duidelijke conclusie, in de vorm van een antwoord op de
%% onderzoeksvra(a)g(en). Wat was jouw bijdrage aan het onderzoeksdomein en
%% hoe biedt dit meerwaarde aan het vakgebied/doelgroep? Reflecteer kritisch
%% over het resultaat. Had je deze uitkomst verwacht? Zijn er zaken die nog
%% niet duidelijk zijn? Heeft het onderzoek geleid tot nieuwe vragen die
%% uitnodigen tot verder onderzoek?

%Uit te werken:
%\begin{itemize}
	%\item Reeds mogelijk om code te delen. Goed voor applicaties met grote domeinlogica, niet goed voor beperkte applicaties, aangezien je meerdere interfaces moet opbouwen.
%	\item Opzet project is moeilijk
%	\item Nog zeer jong, nog heel beperkt
%	\item Weinig documentatie
%	\item Door dit onderzoek is er nu een handleiding om een KN project op te zetten
	%\item Verdere vragen, mogelijkheden in de toekomst om hardware (zoals camera) aan te spreken via KN?
%	\item verder onderzoek: plugin mogelijk dat zelf project opzet? Mogelijkheden bij iOS om framework toe te voegen zonder build phase script?
%\end{itemize}
Uit de proof-of-concept, in sectie \ref{sec:poc}, valt te besluiten dat het maken van een applicatie met Kotlin/Native reeds mogelijk is. Het framework biedt de kans om domeinlogica te delen over de verschillende platformen die ondersteund moeten worden. In vergelijking met een framework zoals React/Native is Kotlin/Native een totaal ander type framework. De focus ligt op het delen van domeinlogica (cross-platform domeinlogica) en niet op het maken van cross-platform applicaties via één codebase. Het gebruiken van Kotlin/Native heeft enkel en alleen maar voordeel indien de applicatie over een grote domeinlogica beschikt. Het framework zal ervoor zorgen dat deze domeinlogica maar één maal moet worden opgebouwd, hiermee kan tijd bespaard worden en er is de mogelijkheid om platformspecifieke verschillen aan te brengen in de code. De user interfaces moeten per platform worden opgebouwd, dit kan zowel positief als negatief zijn. Er is de mogelijkheid om native applicaties te bouwen met een native uiterlijk, maar natuurlijk zijn er de dubbele kosten om een applicatie te bouwen en er is dubbel zoveel tijd nodig.

Daarnaast is de opzet van een Kotlin/Native project nog zeer omslachtig. Momenteel bestaat er geen plugin voor een IDE zoals IntelliJ om een volledig Kotlin/Native project automatisch te laten genereren. Er is kennis nodig van Gradle en de mappenstructuur moet overgenomen worden van de voorbeeldprojecten van JetBrains. Dit geeft aan dat een project niet zomaar wordt opgezet. Er is zo goed als geen officiële documentatie over de opzet en gebruik van Kotlin/Native. De enige bron van informatie zijn enkele ontwikkelaars die Kotlin/Native onderzoeken en gebruiken en de GitHub repository van JetBrains is zeker en vast aan te raden om te raadplegen.

Het framework is nog zeer jong. De huidige versie is 0.6 en de mogelijkheden zijn nog beperkt. Een aantal voorbeelden hiervan zijn:
\begin{itemize}
	\item Er kan in de platformspecifieke iOS Kotlin code geen gebruik gemaakt worden van de iOS protcollen en libraries.
	\item Het aanspreken van hardware via een uniforme manier is nog niet mogelijk.
	\item De mogelijkheden van de gemeenschappelijke code is nog zeer beperkt. Aangezien deze code zowel op iOS en Android moet kunnen draaien. Er kan dus bijvoorbeeld geen interface worden geïmplementeerd, zoals Serializable, aangezien dit niet gekend is in iOS.
\end{itemize}

Maar dit betekent niet dat dit framework nog niet gebruikt kan worden waarvoor het ontwikkeld is. Indien je domeinlogica simpel blijft en er is geen nood aan speciale interfaces (Android) of bepaalde protocollen (iOS), dan volstaan de huidige mogelijkheden van dit framework. Dit framework heeft zeker en vast potentieel maar er moet natuurlijk nog veel veranderd worden. Eerst en vooral is er nood aan een stabiele versie met alle basismogelijkheden. Daarna zal duidelijk worden of dit framework zal aanslaan bij de ontwikkelaars.

De bijdrage van dit onderzoek is een volledige handleiding over het gebruik van Kotlin/Native. De basis over de LLVM compiler en Kotlin/Native komen aan bod, en er is een documentatie over de volledige opzet van een project. Dit was toch het grote gebrek in de Kotlin/Native wereld.

De uitkomst van dit onderzoek was gedeeltelijk verwacht. Er werd verwacht dat de mogelijkheden van dit framework nog meer beperkt waren. Ik had niet verwacht dat er reeds de mogelijkheid was om Kotlin te gebruiken voor iOS development, aangezien Kotlin gebruik maakt van de JVM. Dit heeft het onderzoek zeer interessant gemaakt, iets waarvan je verwacht had dat de mogelijkheden nog zeer beperkt waren maar door ermee te werken constateer je dat er veel meer mogelijk is en je toch al mooie dingen kan maken.

Een verder verloop van dit onderzoek is zeker en vast nodig. Er kan onderzocht worden of er de mogelijkheid is om een plugin te intergreren in een IDE, het aanspreken van hardware mogelijk is, zoals de camera, het gebruik van iOS protocollen en libraries in Kotlin en het importeren van het iOS framework in Xcode zonder een buildscript. Dit zijn momenteel vier belangrijke gebreken die in de toekomst verder onderzocht moeten worden.